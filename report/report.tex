%%%%%%%%%%%%%%%%%%%%%%%%%%%%%%%%%%%%%%%%%
% Short Sectioned Assignment
% LaTeX Template
% Version 1.0 (5/5/12)
%
% This template has been downloaded from:
% http://www.LaTeXTemplates.com
%
% Original author:
% Frits Wenneker (http://www.howtotex.com)
%
% License:
% CC BY-NC-SA 3.0 (http://creativecommons.org/licenses/by-nc-sa/3.0/)
%
%%%%%%%%%%%%%%%%%%%%%%%%%%%%%%%%%%%%%%%%%

%----------------------------------------------------------------------------------------
%	PACKAGES AND OTHER DOCUMENT CONFIGURATIONS
%----------------------------------------------------------------------------------------

\documentclass[paper=a4, fontsize=11pt]{scrartcl} % A4 paper and 11pt font size

\usepackage[T1]{fontenc} % Use 8-bit encoding that has 256 glyphs
\usepackage{fourier} % Use the Adobe Utopia font for the document - comment this line to return to the LaTeX default
\usepackage[english]{babel} % English language/hyphenation
\usepackage{amsmath,amsfonts,amsthm} % Math packages
\usepackage{graphicx}
\usepackage[
backend=biber,
style=alphabetic,
sorting=ynt]
{biblatex}
\addbibresource{sample.bib}

\usepackage{sectsty} % Allows customizing section commands
\allsectionsfont{\centering \normalfont\scshape} % Make all sections centered, the default font and small caps

\usepackage{fancyhdr} % Custom headers and footers
\pagestyle{fancyplain} % Makes all pages in the document conform to the custom headers and footers
\fancyhead[L]{Artificial Intelligence} % No page header - if you want one, create it in the same way as the footers below
\fancyhead[C]{Francesco Berla} % No page header - if you want one, create it in the same way as the footers below
\fancyhead[R]{\today} % No page header - if you want one, create it in the same way as the footers below
\fancyfoot[L]{} % Empty left footer
\fancyfoot[C]{} % Empty center footer
\fancyfoot[R]{\thepage} % Page numbering for right footer
\renewcommand{\headrulewidth}{0pt} % Remove header underlines
\renewcommand{\footrulewidth}{0pt} % Remove footer underlines
\setlength{\headheight}{13.6pt} % Customize the height of the header

\numberwithin{equation}{section} % Number equations within sections (i.e. 1.1, 1.2, 2.1, 2.2 instead of 1, 2, 3, 4)
\numberwithin{figure}{section} % Number figures within sections (i.e. 1.1, 1.2, 2.1, 2.2 instead of 1, 2, 3, 4)
\numberwithin{table}{section} % Number tables within sections (i.e. 1.1, 1.2, 2.1, 2.2 instead of 1, 2, 3, 4)

\setlength\parindent{0pt} % Removes all indentation from paragraphs - comment this line for an assignment with lots of text

%----------------------------------------------------------------------------------------
%	TITLE SECTION
%----------------------------------------------------------------------------------------

\newcommand{\horrule}[1]{\rule{\linewidth}{#1}} % Create horizontal rule command with 1 argument of height

\title{	
\normalfont \normalsize 
\textsc{Universit\`a della Svizzera italiana} \\ [25pt] % Your university, school and/or department name(s)
\horrule{0.5pt} \\[0.4cm] % Thin top horizontal rule
\huge AI Cup Report \\ % The assignment title
\horrule{2pt} \\[0.5cm] % Thick bottom horizontal rule
}

\author{Francesco Berla} % Your name

\date{\normalsize\today} % Today's date or a custom date

\begin{document}

\maketitle % Print the title

%----------------------------------------------------------------------------------------
%	PROBLEM 1
%----------------------------------------------------------------------------------------

\section{Approach}

\subsection{Failed try with Genetic algorithms}

The first approach was using Genetic algorithms, I tried them because they fascinated me.
I implemented the program quite fast but I discovered that the crossover operation wasn't very good, it would rarely produce better children and combined with the fact that it was slow I couldn't get to a nice result in the given time.
I tried to implement many different crossover operations (Ordered Crossover,
PMX crossover \cite{oluk_2002}, \dots) but the gain was not significant.
I played a lot with all the parameters and with the positioning of the local search (optimizing each child, optimizing only the best of each generation, \dots) but none of my efforts came to a good result.
My algorithms was always too slow to converge and my solution was bad in the time frame that was given for solving the problem.
The code of my implementation of a genetic algorithm is in the file \emph{genetic.py}.

%------------------------------------------------

\subsubsection{Iterative Local Search}

After the failure with the Genetic algorithms I had to go back to the \emph{Iterative local search}.
My proposed solution is based on the work of Umberto Junior Mele that he has given as foundation.
It's simply an Iterative Local Search using the simulated annealing as base.
The randomization for escaping the local optimum is created by a random swap of two nodes.
The local search algorithm is 2opt.
This has been chosen in order to do more iterations as possible.
Using a Lin-Kernigan, or 3opt, approach would have produced better results but in a lot more time and this could have been negative considering the 3 minutes time frame.

\subsection{Improved attempt with Cython}

My Iterative Local Search still performed in a unsatisfactory way, therefore I
tried to improve the running time using Cython.
Cython has the advantage that it works directly with Python code without big
modifications and it promises significant speedups.
After the initial setup (see the \emph{setup.py} and all the files extensions
changed to \emph{.pyx}) I ran all the algorithm noticing a some speed-up but
nothing completely game-breaking.
One of the possible reason is the unoptimized integration with numpy arrays,
another one is that I am not using Cython correctly in order to leverage its
power.

%----------------------------------------------------------------------------------------
%	PROBLEM 2
%----------------------------------------------------------------------------------------

\section{System}

The system used is Arch Linux (kernel: Linux-hardened 5.3.13.a-1), python version 3.8.0.

\section{Usage}

Execute 

\texttt{python runniamo.py ProblemName}.

If you want to modify something, or if you are running for the first time, for
compiling you have to type 

\texttt{python setup.py build\_ext --inplace}.

%------------------------------------------------

\section{literature}

In the folder \emph{papers} you will find some of the publications that I used
to build my work.
The majority are related to genetic algorithms.
In addition to those scientific papers I visited many websites where people
were presenting their ideas about possible optimizations and parameters.

\printbibliography

\end{document}
